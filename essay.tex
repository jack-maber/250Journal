% Please do not change the document class
\documentclass{scrartcl}
% Please do not change these packages
\usepackage[hidelinks]{hyperref}
\usepackage[none]{hyphenat}
\usepackage{setspace}
\doublespace
% You may add additional packages here
\usepackage{amsmath}
\usepackage{url}
\makeatletter
\g@addto@macro{\UrlBreaks}{\UrlOrds}
\makeatother
% Please include a clear, concise, and descriptive title
\title{210 Research Journal}
% Please do not change the subtitle
\subtitle{COMP210 - Research Journal}
% Please put your student number in the author field
\author{1606119}
\begin{document}
\maketitle
Paper 1 \cite{mentzelopoulos2015hardware} evaluates the available Hardware interfaces that can be utilized for VR experiences, and the effectiveness of them to keep players immersed in said experience, as one of the most important parts of a game is the experience that the player takes away from it, with the paper stating that ``The player will remember a game because of the adventure and the story lived.''\cite{mentzelopoulos2015hardware}, so having a way of interacting with the game that can be used completely spontaneously without having to explain how it works to the user is paramount in creating a good game experience.\\

To test out if the current Hardware Interfaces were the best to create an ``Immersive'' experience, the creators of this paper created two prototypes best suited to each of the devices they picked (Razer Hydra, Xbox Controller), with a Puzzle type game and an action game respectively, both of the prototypes also featured ``game design narratives to provide a relative physical bond between the player and the game by keeping the game interest and player's emotions in a high standard.''\cite{mentzelopoulos2015hardware}, as to keep the player immersed and motivated to progress in the game, which is a topic that a few other of the papers mentioned, especially paper 2\cite{hvannberg2012exploitation}, where it is stated that ``motivation was an influential psychological factor that is positively related to the VR measurement outcomes.''\cite{hvannberg2012exploitation}, and as the creators of Paper 1 are carrying out research, providing the player with said motivation in the way of a narrative is very much required as the test base that they were using were new to both prototypes, and thus needed a narrative to guide them through the game and adapt to both of the control schemes, which would improve the end results of the research over if they had just dropped them into a scene and gave them no direction. The results of the research found that the Xbox controller was easier to adapt to earlier on in the study, which makes perfect sense it was carried out on students with previous gaming experience, so they would be used to using such a common input method, but after the tutorial the ``overall complexity of adapting in both controllers was ambient.'' \cite{mentzelopoulos2015hardware}, although it was noted further on that players found the puzzle game more enjoyable due to the use of the Hydra controllers, as well as less players experiencing motion sickness, which the creators put down to Dreams ``using a better interface for the user to interact with the hydra controllers.''\cite{mentzelopoulos2015hardware}\\

Paper 2\cite{hvannberg2012exploitation} delves deeper into the heuristics that can be used to evaluate ``Virtual Environments'', with the overarching theme being that the vast majority of research papers will pick and choose the heuristics that are applicable to the research they are undertaking, the most referenced being Sutcliffe and Gault, which were created in 2004 and as VR has become much more accessible since then, use of that certain set of heuristics has risen, and although they are widely used, ``only one paper used them fully''\cite{hvannberg2012exploitation}, with some only using a few, while I found in my research that these set Heuristics were hardly if ever referenced, with paper 3\cite{venta2014investigating} creating their own study questions, however as this paper was investigating Mobile Mixed Reality (MMR) and it's usability compared to normal AR and VR applications, which is a relatively un-researched area, especially when comparing normal AR and Augmented Virtuality, with research into this particular field being ``non-existent''\cite{venta2014investigating}, it would have been hard to gather usable data if they had utilized existing heuristic frameworks, an additional source also points out that only ``some of the twelve heuristics apply to AR systems with slight adaptations, but others do not.``\cite{furht2011handbook} ,however by creating their own set of user tests, they gained much better insight into AR and MMR, finding that hybrids of existing systems would be the most immersive and usable option.\\

Paper 4\cite{won2015homuncular} moves away from the theory of VR, and onto the people who use it, looking at our innate human ability to adapt to bodies that unlike our own, which, through the use of VR and related systems, is very easy, this study carried out 3 tests to see how well we can adapt to bodies unlike ours and perform tasks, as to see how well we can rewrite our ``Homonculus'', the map of our body in our brain, this ``map'' is flexible and can be adapted. This paper tests this out by utilising VR, and placing the testers into bodies unlike their own, for example, giving them a 3rd arm that extended out of their chest which was controlled by the rotation of the users wrist, and although the movement is controlled in a different modality, users were able to adapt to it and utilize it effectively, even when ``the long third limb looked radically different from the avatar in the normal condition.''\cite{won2015homuncular}, which is explained by Steptoe\cite{steptoe2013human} where users gain a sense of ownership and thus agency over an extra limb, in this case a tail, if it affected by the users movement, however if it doesn't respond to user input, the feeling of ownership is lessened. This adaptation can also work the other way, in the case of amputees who lose limbs, the mind can remap the feeling they would have felt to ``other parts of the body''\cite{won2015homuncular}, this can then lead to a phenomenon called Phantom Limb Pain, where amputees experience sensation perceived where their limb used to be, although this can normally be solved by utilising mirrors to complete therapy, research has been carried out into utilising Multimodal VR to create a virtual limb and map it to movements given to it by the users, with up to an 86 percent decrease in pain\cite{wake2015multimodal}, it shows how versatile VR can be.\\

Paper 5 \cite{rosa2016re} also delves into the use of Multimodal techniques, but this time in AR, where it investigates the use of different ``modalities'' to enhance AR and create a Mixed Reality, which was touched on earlier in paper 3. Multimodal in essence is removing the virtual overlay from the ``real'' base and replacing it with other ``virtual'' augmentations, one example that stood out to me was ``the directions are presented as vibrations around the waist ...the visuals are real and the directions are virtual''\cite{rosa2016re}, showing an effective method of removing the directional overlay from the ``real'' and placing it into another modality, as moving information from the ``real'' overlay may reduce stress and the overall comfort of the user, as is less to look at, which I can defientely understand. Paper 5 also looks at the risk of virtual modification, and how our perception of the real world vs the virtual world may change, something that paper 6\cite{madary2016real} features heavily, along with the fact that VR can induce such strong feelings of embodiment, as seen with the point about the tail earlier, these feelings of embodiment can be increased by adding in such things as eye and face tracking\cite{collingwoode2017effect};the most interesting point that the paper brought up was that full body tracking, as used in paper 4, could lead to massive security risks as users are giving away their ``Kinematic Fingerprint'', as nobody moves the exact same, so researchers and developers of such VR experiences will have to be very clear with how they use all of the data they gather. 







\noindent 




\raggedright
\bibliographystyle{ieeetran}
\bibliography{references}
\end{document}